\documentclass[a4paper]{report}
\usepackage{amsmath}
\usepackage{amsthm}
\newcommand{\vect}{\mathbf}
\begin{document}

\chapter{The Rendering Pipeline}

\chapter{Vectors}
% Exercises for Chapter 2
\begin{enumerate}
	\item If $\vect{P} = \langle 2,2,1 \rangle$ and $\vect{Q} = \langle 1,-2,0 \rangle$, then
	\begin{enumerate}
		\item $\vect{P} \cdot \vect{Q} = 0$
		\item $\vect{P} \times \vect{Q} = \langle 0,1,-6 \rangle$
		\item ${\text{proj}}_{\vect{P}}\vect{Q} = \langle -\frac{4}{9}, -\frac{4}{9}, -\frac{2}{9} \rangle$
	\end{enumerate}
	\item
		\begin{align}
			\mathbf{e}_1' &= \langle \frac{\sqrt{2}}{2}, \frac{\sqrt{2}}{2}, 0 \rangle\\
			\mathbf{e}_2' &= \langle -1, 1, -1 \rangle\\
			\mathbf{e}_3' &= \langle 1, -1, -2 \rangle
		\end{align}
	\item $17.5$
	\item
		\begin{proof}
			\begin{align*}
				(\vect{V} \cdot \vect{W})^2 + \lVert \vect{V} \times \vect{W} \rVert^2
				&= \cos^2 \alpha \lVert \vect{V} \rVert^2 \lVert \vect{W} \rVert^2
				+ \sin^2 \alpha \lVert \vect{V} \rVert^2 \lVert \vect{W} \rVert^2\\
				&= (\cos^2 \alpha + \sin^2 \alpha)(\lVert \vect{V} \rVert^2 \lVert \vect{W} \rVert^2)\\
				&= V^2W^2
			\end{align*}
		\end{proof}
	\item
		\begin{proof}
			Direct computation of the $x$ component gives us
			\begin{equation*}
				(\vect{P} \times \vect{Q} \times \vect{R})_x = \vect{P}_z \vect{Q}_x \vect{R}_z - \vect{P}_x \vect{Q}_z \vect{R}_z - \vect{P}_x \vect{Q}_y \vect{R}_y + \vect{P}_y \vect{Q}_x \vect{R}_y
			\end{equation*}
			\begin{align*}
				((\vect{P} \cdot \vect{R}) \vect{Q} - (\vect{Q} \cdot \vect{R}) \vect{P})_x
				&= \vect{P}_x \vect{Q}_x \vect{R}_x + \vect{P}_y \vect{Q}_x \vect{R}_y + \vect{P}_z \vect{Q}_x \vect{R}_z - \vect{P}_x \vect{Q}_x \vect{R}_x - \vect{P}_x \vect{Q}_y \vect{R}_y - \vect{P}_x \vect{Q}_z \vect{R}_z\\
				&= \vect{P}_z \vect{Q}_x \vect{R}_z - \vect{P}_x \vect{Q}_z \vect{R}_z - \vect{P}_x \vect{Q}_y \vect{R}_y + \vect{P}_y \vect{Q}_x \vect{R}_y,
			\end{align*}
			The $y$ and $z$ components can be checked in a similar manner.
		\end{proof}
	\item
		\begin{proof}
			\begin{align*}
				\lVert \vect{P} - \vect{Q} \rVert^2
				&= (\vect{P} - \vect{Q}) \cdot (\vect{P} - \vect{Q})\\
				&= P^2 + Q^2 - 2 \vect{P} \cdot \vect{Q}\\
				&\geq P^2 + Q^2 - 2 \lVert \vect{P} \rVert \lVert \vect{Q} \rVert\\
				&= (\lVert \vect{P} \rVert - \lVert \vect{Q} \rVert)^2
			\end{align*}
		\end{proof}
\end{enumerate}

\chapter{Matrices}
% Exercises for Chapter 3
\begin{enumerate}
	\item
	\begin{enumerate}
		\item
			\begin{equation*}
				\begin{vmatrix}
					2 & 7\\
					-3 & \frac{1}{2}
				\end{vmatrix}
				=
				22
			\end{equation*}
		\item
			\begin{equation*}
				\begin{vmatrix}
					0 & 0 & 1\\
					0 & 1 & 0\\
					1 & 0 & 0
				\end{vmatrix}
				=
				-1
			\end{equation*}
		\item
			\begin{equation*}
				\begin{vmatrix}
					\frac{1}{2} & \frac{\sqrt{3}}{2} & 0\\
					-\frac{\sqrt{3}}{2} & \frac{1}{2} & 0\\
					0 & 0 & 1
				\end{vmatrix}
				=
				1
			\end{equation*}
		\item
			\begin{equation*}
				\begin{vmatrix}
					5 & 7 & 1\\
					17 & 2 & 64\\
					10 & 14 & 2
				\end{vmatrix}
				=
				0
			\end{equation*}
	\end{enumerate}
\item
	\begin{enumerate}
		\item
			\begin{equation*}
				\begin{vmatrix}
					2 & 0 & 0\\
					0 & 3 & 0\\
					0 & 0 & 4
				\end{vmatrix}^{-1}
				=
				\begin{vmatrix}
					\frac{1}{2} & 0 & 0\\
					0 & \frac{1}{3} & 0\\
					0 & 0 & \frac{1}{4}
				\end{vmatrix}
			\end{equation*}
		\item
			\begin{equation*}
				\begin{vmatrix}
					1 & 0 & 0\\
					0 & 2 & 2\\
					3 & 0 & 8
				\end{vmatrix}^{-1}
				=
				\begin{vmatrix}
					1 & 0 & 0\\
					\frac{3}{8} & \frac{1}{2} & -\frac{1}{8}\\
					-\frac{3}{8} & 0 & \frac{1}{8}
				\end{vmatrix}
			\end{equation*}
		\item
			\begin{equation*}
				\begin{vmatrix}
					\cos \theta & 0 & -\sin \theta\\
					0 & 1 & 0\\
					\sin \theta & 0 & \cos \theta
				\end{vmatrix}^{-1}
				=
				\begin{vmatrix}
					\cos \theta & 0 & -\sin \theta\\
					0 & 1 & 0\\
					\sin \theta & 0 & \cos \theta
				\end{vmatrix}
			\end{equation*}
		\item
			\begin{equation*}
				\begin{vmatrix}
					1 & 0 & 0 & 4\\
					0 & 1 & 0 & 3\\
					0 & 0 & 1 & 7\\
					0 & 0 & 0 & 1
				\end{vmatrix}^{-1}
				=
				\begin{vmatrix}
					1 & 0 & 0 & -4\\
					0 & 1 & 0 & -3\\
					0 & 0 & 1 & -7\\
					0 & 0 & 0 & 1
				\end{vmatrix}
			\end{equation*}
	\end{enumerate}
\item
	\begin{equation*}
		\left[
		\begin{matrix}
			x\\
			y\\
			z
		\end{matrix}
		\right]
		=
		a
		\left[
		\begin{matrix}
			1\\
			-2\\
			1
		\end{matrix}
		\right]
	\end{equation*}
\item $\lambda_1=2$, $\lambda_2=-1$, $\lambda_3=5$.
\item TODO
\item \begin{proof}
		We can use induction on $n$, where $n$ is the size of the matrix $M$, we want to prove that $\det \mathbf{M} = \prod_{i=1}^{n}M_{ii}$. For the base case $n = 1$ we have $\det \mathbf{M} = M_{11}$.
		Assume the case holds for $\mathbf{N}$ of size $k$, show it holds for $\mathbf{M}$ of size $k + 1$.
		\begin{align*}
			\begin{vmatrix}
				M_{11} & \vect{V}^T\\
				\mathbf{0} & \mathbf{N}
			\end{vmatrix}
			&=
			M_{ii} \det \mathbf{N}\\
			&=
			\prod_{i=1}^{k+1}M_{ii}
		\end{align*}
\end{proof}
\item TODO
\end{enumerate}

\chapter{Transforms}
% Exercises for Chapter 4
\begin{enumerate}
	\item
		\begin{align*}
			\mathbf{R}_x &=
			\left[ \begin{matrix}
				1 & 0 & 0\\
				0 & \frac{\sqrt 3}{2} & -\frac{1}{2}\\
				0 & \frac{1}{2} & \frac{\sqrt 3}{2}
			\end{matrix} \right],\\
			\mathbf{R}_y &=
			\left[ \begin{matrix}
				\frac{\sqrt 3}{2} & 0 & \frac{1}{2}\\
				0 & 1 & 0\\
				-\frac{1}{2} & 0 & \frac{\sqrt 3}{2}
			\end{matrix} \right],\\
			\mathbf{R}_z &=
			\left[ \begin{matrix}
				\frac{\sqrt 3}{2} & -\frac{1}{2} & 0\\
				\frac{1}{2} & \frac{\sqrt 3}{2} & 0\\
				0 & 0 & 1
			\end{matrix} \right]
		\end{align*}
	\item $\mathbf{q} = \pm (\frac{\sqrt 3}{2} + \langle 0, \frac{3}{10}, \frac{2}{5} \rangle$)
	\item If we add the following results we get to the desired result
		\begin{align*}
			s_1s_2 &= w_1w_2\\
			\mathbf{v}_1 \cdot \mathbf{v}_2 &= x_1x_2 + y_1y_2 + z_1z_2\\
			s_1\mathbf{v}_2 &= \langle w_1x_2 + w_1y_2 + w_1z_2 \rangle\\
			s_2\mathbf{v}_1 &= \langle w_2x_1 + w_2y_1 + w_2z_1 \rangle\\
			\mathbf{v}_1 \times \mathbf{v}_2 &= \langle y_1z_2 - z_1y_2, z_1x_2 - z_2x_1, x_1y_2 - y_1x_2 \rangle.
		\end{align*}
	\item TODO
\end{enumerate}

\end{document}
